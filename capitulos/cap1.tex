% Revisado por Gabriel Saraiva - 14:02

\chapter{Introdução}
\label{cap1}

    A rápida evolução da computação, devido a concorrência entre os fabricantes e sua demanda perene, possibilitou o barateamento e o aumento na capacidade dos sistemas computacionais. Resultado de grandes avanços tecnológicos, dos microprocessadores aliado aos sistemas de conexão e redes, a computação se tornou onipresente em muitos campos de nossa sociedade, como destacado por ~\cite{tanenbaum}.

    A evolução desses sistemas proporcionou a solução de problemas maiores e mais complexos, que comumente não são possíveis de serem solucionados com dispositivos isolados. Para a solução desses problemas são utilizados sistemas distribuídos, onde sistemas computacionais diversos cooperam, se comunicando via redes de computadores e internet para atingir um objetivo comum ~\cite{coulouris}.

    Dentro da computação distribuída, como é necessário armazenar esses dados e fazer o compartilhamento deles, existem os sistemas de arquivos distribuídos, que são responsáveis por fornecer acesso aos dados de forma constante e confiável. Nesse trabalho será tratado o FlexA (\textit{Flexible and Adaptable Distributed File System}), desenvolvido no GSPD (Grupo de Sistemas Paralelos e Distribuídos) da UNESP.


\section{Motivação}

    Dentre os fatores que motivam esse trabalho estão a necessidade de tornar o FlexA um sistema mais fácil de ser expandido no futuro, através de  interfaces que tornem o projeto mais aberto e flexível com relação ao desenvolvimento e arquitetura. Também aproveitando a grande popularização de dispositivos de computação móvel como por exemplo \textit{tablets} e \textit{smartphones} que com a presença e uso constante desses dispositivos o FlexA poderia ser utilizado pelo usuário de forma prática e rápida.

\section{Objetivos}

    Os objetivos gerais desse trabalho são adaptar as interfaces de comunicação do FlexA, de modo a remodelar interfaces de comunicação entre cliente e servidor, e implementar uma versão do módulo cliente do sistema para dispositivos móveis que utilizem Android.

    Os objetivos específicos são a implementação das operações desconectadas que são responsáveis por gerenciar o FlexA enquanto estiver desconectado e sincronizá-lo ao conectar novamente com os servidores; implementação de um módulo de controle de falhas de conexão, e mecanismos mínimos de segurança dos dados e da comunicação com os servidores, utilização de criptografia e chaves de identificação. Como esses dispositivos possuem uma capacidade de armazenamento menor, existe a necessidade de um melhor gerenciamento dos arquivos residentes na memória cache do sistema de arquivo distribuído (SAD).

\section{Organização do texto}

    No trabalho estão, além dessa introdução, no capítulo \ref{cap2} a fundamentação teórica sobre a qual o trabalho foi desenvolvido, com o estudo de sistemas distribuídos, sistemas de arquivos distribuídos, chamada de procedimentos remotos e também sobre Android e dispositivos móveis. No capítulo \ref{cap3} é descrito o desenvolvimento do projeto, são mostrados os diversos mecanismos utilizados e implementados. O capítulo \ref{cap4} apresenta os cenários de testes utilizados, e os testes realizados para validação do projeto. Por fim, no capítulo \ref{cap5} são feitos comentários sobre a execução do projeto desenvolvido e dos testes realizados, e são apresentadas as dificuldades e novos caminhos para projetos futuros.
    