\chapter{Introdução}

A rápida evolução da computação, devido a concorrência entre os fabricantes e sua demanda perene, possibilitou o barateamento e o aumento na capacidade dos sistemas computacionais. Resultado de grandes avanços tecnológicos, dos microprocessadores aliado aos sistemas de conexão e redes, a computação se tornou em muitos campos de nossa sociedade onipresente, como destacado por ~\cite{coulouris}.

A evolução desses sistemas, proporcionou a solução de problemas maiores e mais complexos, que comumente não são possíveis de serem solucionados com dispositivos isolados. Para a solução desses problemas são utilizados sistemas distribuídos, onde sistemas computacionais diversos cooperam, se comunicando via redes de computadores e internet para atingir um objetivo comum ~\cite{tanenbaum}.

E dentro da computação distribuída, existem os sistemas de arquivos distribuídos, cujo objetivo é fornecer meios simples e eficientes para gerenciar grandes quantidades de arquivos, fornecendo acesso


E dentro desse campo de estudo, existem diversos Sistemas de Arquivos Distribuí-
dos, com diferentes enfoques e características desejáveis. Dessa maneira, verificou-
se a necessidade de agregar os principais fatores positivos (escalabilidade, desempe-
nho, segurança, disponibilidade, etc) dos principais Sistemas de Arquivos Distribuídos
no FlexA (Flexible and Adaptable Distributed File System), desenvolvido no GSPD
(Grupo de Sistemas Paralelos e Distribuídos).
