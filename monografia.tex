%% Revisado por Gabriel Saraiva
%% Revisado por Renata

%% Copyright 2012-2014 by abnTeX2 group at http://abntex2.googlecode.com/ 
% ------------------------------------------------------------------------
% ------------------------------------------------------------------------
% abnTeX2: Modelo de Trabalho Acadêmico (tese de doutorado, dissertacao de
% mestrado e trabalhos monográficos em geral) em conformidade com 
% ABNT NBR 14724:2011: Informação e documentação - Trabalhos acadêmicos -
% Apresentação
% ------------------------------------------------------------------------
% ------------------------------------------------------------------------

\documentclass[
	% -- opções da classe memoir --
	12pt,				% tamanho da fonte
	openright,			% capítulos começam em pág ímpar (insere página vazia caso preciso)
	oneside,			% para impressão em verso e anverso. Oposto a oneside
	a4paper,			% tamanho do papel. 
	% -- opções da classe abntex2 --
	%chapter=TITLE,		% títulos de capítulos convertidos em letras maiúsculas
	%section=TITLE,		% títulos de seções convertidos em letras maiúsculas
	%subsection=TITLE,	% títulos de subseções convertidos em letras maiúsculas
	%subsubsection=TITLE,% títulos de subsubseções convertidos em letras maiúsculas
	% -- opções do pacote babel --
	english,			% idioma adicional para hifenização
	brazil				% o último idioma é o principal do documento
	]{abntex2}

% ---
% Pacotes
% ---
\usepackage{lmodern}			% Usa a fonte Latin Modern			
\usepackage[T1]{fontenc}		% Seleção de códigos de fonte.
\usepackage[utf8]{inputenc}		% Codificação do documento (conversão automática dos acentos)
\usepackage{lastpage}			% Usado pela Ficha catalográfica
\usepackage{indentfirst}		% In denta o primeiro parágrafo de cada seção.
\usepackage{color}				% Controle das cores
\usepackage{graphicx}			% Inclusão de gráficos
\usepackage{microtype} 			% para melhorias de justificação
\usepackage{comment}           %para inserir blocos de comentário
\usepackage{mathptmx}
\usepackage{amsmath}
\usepackage{mathtools}
\usepackage{empheq}
\usepackage[brazilian,hyperpageref]{backref}	 % Paginas com as citações na bibl
\usepackage[alf]{abntex2cite}	% Citações padrão ABNT
\usepackage{etoolbox}

\usepackage{float}


% --- 
% CONFIGURAÇÕES DE PACOTES
% --- 

\floatstyle{plaintop}
\restylefloat{table}

\graphicspath{ {img/} } % Local das imagens
\apptocmd{\sloppy}{\hbadness 10000\relax}{}{} %



% ---
% Macros e funções
% ---

% Utilizado para por bordas nos protocolos
\newcommand*\widefbox[1]{\fbox{\hspace{2em}#1\hspace{2em}}}

\newcommand{\bordaProtocolo}[1]{
    \centering
    \framebox{\parbox{\dimexpr\linewidth-15\fboxsep-2\fboxrule}{\itshape%
        \vspace{2mm}
        #1
        \vspace{2mm}
    }}
}


% Configurações do pacote backref
% Usado sem a opção hyperpageref de backref
\renewcommand{\backrefpagesname}{Citado na(s) página(s):~}
% Texto padrão antes do número das páginas
\renewcommand{\backref}{}
% Define os textos da citação
\renewcommand*{\backrefalt}[4]{
	\ifcase #1 %
		Nenhuma citação no texto.%
	\or
		Citado na página #2.%
	\else
		Citado #1 vezes nas páginas #2.%
	\fi}%
% ---


%%%%%%%%%%%%%%%%%%%%%%%%%%%%%%%%%%%%%%%%%%%%%%%%%%%%%%%%%%%%%%%%%%%%%%%%%%%%%%%%%%%%%%%%%%%%%
%                                                                                           %
%                           Construindo elementos pré textuais                              %
%                                                                                           %
%%%%%%%%%%%%%%%%%%%%%%%%%%%%%%%%%%%%%%%%%%%%%%%%%%%%%%%%%%%%%%%%%%%%%%%%%%%%%%%%%%%%%%%%%%%%%

% ---
% Informações de dados para CAPA e FOLHA DE ROSTO
% ---

\titulo{Implementação e Adaptação de um Sistema de Arquivos Distribuídos Móvel Adaptável e Flexível para Dispositivos Móveis}
\autor{Gabriel Henrique Martinez Saraiva}
\local{São José do Rio Preto}
\data{2014}
\orientador[Profa. Dra.]{Renata Spolon Lobato}

%%inserir caso ouver%% -> \coorientador{}


\instituicao{%
  Universidade Estadual Paulista "Júlio de Mesquita Filho" (UNESP)
  \par
  Instituto de Biociências, Letras e Ciências Exatas (IBILCE)
  \par
  Bacharelado em Ciência da Computação}
  
\tipotrabalho{Monografia de Qualificação}
% O preambulo deve conter o tipo do trabalho, o objetivo, 
% o nome da instituição e a área de concentração 
\preambulo{Monografia apresentada ao Departamento de Ciências de  Computação  e  Estatística  do  Instituto  de Biociências,  Letras  e  Ciências  Exatas  da Universidade  Estadual  Paulista  ”Júlio  de  Mesquita Filho”,  como  parte  dos  requisitos  necessários  para obtenção do título de "Bacharel" em Ciência  da Computação da UNESP}
% ---

% informações do PDF
\makeatletter
\hypersetup{
     	%pagebackref=true,
		pdftitle={\@title}, 
		pdfauthor={\@author},
    	pdfsubject={\imprimirpreambulo},
	    pdfcreator={LaTeX with abnTeX2},
		pdfkeywords={abnt}{latex}{abntex}{abntex2}{trabalho acadêmico}, 
		colorlinks=true,       		% false: boxed links; true: colored links
    	linkcolor=blue,          	% color of internal links
    	citecolor=blue,        		% color of links to bibliography
    	filecolor=magenta,      		% color of file links
		urlcolor=blue,
		bookmarksdepth=4
}
\makeatother
% --- 


% ---
% Configurações de aparência do PDF final

% alterando o aspecto da cor azul
\definecolor{blue}{RGB}{41,40,40}

% --- 
% Espaçamentos entre linhas e parágrafos 
% --- 

% O tamanho do parágrafo é dado por:
\setlength{\parindent}{1.3cm}

% Controle do espaçamento entre um parágrafo e outro:
\setlength{\parskip}{0.2cm}  % tente também \onelineskip

% ---
% compila o indice
% ---
\makeindex
% ---


% ----
% Início do documento
% ----
\begin{document}

% Retira espaço extra obsoleto entre as frases.
\frenchspacing 

% ----------------------------------------------------------
% ELEMENTOS PRÉ-TEXTUAIS
% ----------------------------------------------------------
% \pretextual

% ---
% Capa
% ---
\imprimircapa
% ---

% ---
% Folha de rosto
% (o * indica que haverá a ficha bibliográfica)
% ---
\imprimirfolhaderosto
% ---

% ---
% Inserir a ficha bibliográfica
% ---

% Isto é um exemplo de Ficha Catalográfica, ou ``Dados internacionais de
% catalogação-na-publicação''. Você pode utilizar este modelo como referência. 
% Porém, provavelmente a biblioteca da sua universidade lhe fornecerá um PDF
% com a ficha catalográfica definitiva após a defesa do trabalho. Quando estiver
% com o documento, salve-o como PDF no diretório do seu projeto e substitua todo
% o conteúdo de implementação deste arquivo pelo comando abaixo:
%
% \begin{fichacatalografica}
%     \includepdf{fig_ficha_catalografica.pdf}
% \end{fichacatalografica}

%%%%%%%%%%%%%% Qualificação não é catalogada %%%%%%%%%%%%%%%%%%%%%%%%%%%%%%%%%%%%%%%%


%
%\begin{comment}
%
%\begin{fichacatalografica}
	%\vspace*{\fill}					% Posição vertical
	%\hrule							% Linha horizontal
	%\begin{center}					% Minipage Centralizado
	%\begin{minipage}[c]{12.5cm}		% Largura
%	
	%\imprimirautor
%	
	%\hspace{0.5cm} \imprimirtitulo  / \imprimirautor. --
%	\imprimirlocal, \imprimirdata-
%	
%	\hspace{0.5cm} \pageref{LastPage} p. : il. (algumas color.) ; 30 cm.\\
%	
%	\hspace{0.5cm} \imprimirorientadorRotulo~\imprimirorientador\\
%	
%	\hspace{0.5cm}
%	\parbox[t]{\textwidth}{\imprimirtipotrabalho~--~\imprimirinstituicao,
%	\imprimirdata.}\\
%	
%	\hspace{0.5cm}
%		1. Palavra-chave1.
%		2. Palavra-chave2.
%		I. Orientador.
%		II. Universidade xxx.
%		III. Faculdade de xxx.
%		IV. Título\\ 			
%	
%	\hspace{8.75cm} CDU 02:141:005.7\\
%	
%	\end{minipage}
%	\end{center}
%	\hrule
%\end{fichacatalografica}

%\end{comment}
% ---

% ---
% Inserir errata
% ---

%%%%%%%%%%%%%%%%%%%%%%%%%%%%%%%%%%% sem errata por enquanto %%%%%%%%%%%%%%%%%%%%%%%%%%%%%%%
\begin{comment}

\begin{errata}

\end{errata}

\end{comment}
% ---

% ---
% Inserir folha de aprovação
% ---

% Isto é um exemplo de Folha de aprovação, elemento obrigatório da NBR
% 14724/2011 (seção 4.2.1.3). Você pode utilizar este modelo até a aprovação
% do trabalho. Após isso, substitua todo o conteúdo deste arquivo por uma
% imagem da página assinada pela banca com o comando abaixo:
%
% \includepdf{folhadeaprovacao_final.pdf}
%

%%%%%%%%%%%%%%%%%%%%%%%% sem aprovação por enquanto %%%%%%%%%%%%%%%%%%%%%%%%%%%%%%%%%%%%%
\begin{comment}

\begin{folhadeaprovacao}

  \begin{center}
    {\ABNTEXchapterfont\large\imprimirautor}

    \vspace*{\fill}\vspace*{\fill}
    \begin{center}
      \ABNTEXchapterfont\bfseries\Large\imprimirtitulo
    \end{center}
    \vspace*{\fill}
    
    \hspace{.45\textwidth}
    \begin{minipage}{.5\textwidth}
        \imprimirpreambulo
    \end{minipage}%
    \vspace*{\fill}
   \end{center}
        
   Trabalho aprovado. \imprimirlocal, 24 de novembro de 2012:

   \assinatura{\textbf{\imprimirorientador} \\ Orientador} 
   \assinatura{\textbf{Professor} \\ Dr. Aleardo Manacero Jr.}
   \assinatura{\textbf{Professora} \\ Dra. Adriana Adriana Barbosa Santos}
  
      
   \begin{center}
    \vspace*{0.5cm}
    {\large\imprimirlocal}
    \par
    {\large\imprimirdata}
    \vspace*{1cm}
  \end{center}
  
\end{folhadeaprovacao}
% ---
\end{comment}

% ---
% Dedicatória
% ---
\begin{dedicatoria}
   \vspace*{\fill}
   \centering
   \noindent
   \textit{ Dedico esse trabalho a todos que algum dia, de alguma forma colaboraram para que se tornasse realidade.} \vspace*{\fill}
\end{dedicatoria}
% ---

% ---
% Agradecimentos
% ---
%%%%%%%%%%%%%%%%%%%%%% agradecimentos %%%%%%%%%%%%%%%%%%%%%%%%%%%%%%%%%
\begin{comment}

\begin{agradecimentos}
À minha mãe pela paciência e amor durante todos esses anos.

\end{agradecimentos}

\end{comment}

% ---

% ---
% Epígrafe
% ---
\begin{epigrafe}
    \vspace*{\fill}
	\begin{flushright}
		\textit{"E quantas vezes você disse que não aguentava mais,\\ mas continuou seguindo em frente?"} \\
		-autor desconhecido
	\end{flushright}
\end{epigrafe}
% ---

% ---
% RESUMOS
% ---

% resumo em português
\setlength{\absparsep}{18pt} % ajusta o espaçamento dos parágrafos do resumo
\begin{resumo}
 
    Nesse trabalho é feito um levantamento bibliográfico sobre sistemas distribuídos, sistemas de arquivos distribuídos, Android e dispositivos móveis e sobre o uso de chamada de procedimentos remotos. 
 
    Servindo como fundamentação teoria, o levantamento bibliográfico embasa a execução do trabalho, que possui duas partes principais, sendo a primeira parte do trabalho sobre a padronização do FlexA (\textit{Flexible and Adaptable Distributed File System}), desenvolvido no Grupo de Sistemas Paralelos e Distribuídos (GSPD), desenvolvendo uma documentação coesa baseada em diagramas de classes, casos de uso funcionais, e protocolos de comunicação. Essa documentação tem como função permitir o progresso futuro do projeto. E na segunda parte estão a adaptação do módulo servidor para que esse seja compatível com as interfaces de comunicação e protocolos propostos, além da implementação de um cliente Android para que o sistema possa ser utilizado de maneira mais fácil e dinâmica pelos usuários.

 \textbf{Palavras-chaves}: Android, chamada de procedimentos remotos, sistemas distribuídos, sistemas de arquivos distribuídos.
\end{resumo}

% resumo em inglês
\begin{resumo}[Abstract]

   This work is a bibliographical survey on distributed systems, distributed file systems, mobile devices and Android, and use of remote procedure calls (RPC).

Serving as theoric foundation, this survey underlies the work execution, which is basically focused in two big sections, the first one being about the standartization of FlexA (Flexible and Adaptable Distributed File System), developed at GSPD (Parallel and Distributed Systems Research Group), developing a cohesive documentation based on class diagrams, functional use cases and communication protocols. The role of this documentation is to allow further progress of the project. And in the second part are: the server module adaptation, so that it becomes compatible with communication interfaces and with the proposed protocols; and the implementation of an Android client so that the system can be dinamically and easily utilized by users.

\textbf{keywords}: Android, remote procedures call, distributed systems, distributed file systems.
\end{resumo}


% ---
% inserir lista de ilustrações
% ---
\pdfbookmark[0]{\listfigurename}{lof}
\listoffigures*
\cleardoublepage
% ---
% inserir lista de tabelas
% ---
\pdfbookmark[0]{\listtablename}{lot}
\listoftables*
\cleardoublepage
% ---

% ---
% inserir lista de abreviaturas e siglas
% ---
\begin{siglas}
    \item $ $
    \begin{tabular}{l cl}
 ACL & - & \textit{Access Control List}\\ 
 AFS & - & \textit{Andrew File System}\\ 
 EXT2 & - & \textit{second extended filesystem}\\ 
 EXT3 & - & \textit{third extended filesystem}\\ 
 EXT4 & - & \textit{fourth extended filesystem}\\ 
 FAT & - & \textit{File Alocation Table}\\ 
 GSPD & - & Grupo de Sistemas Paralelos e Distribuídos \\
 MD5 & - & \textit{Message-Digest algorithm 5}\\ 
 NFS & - & \textit{Network File System} \\ 
 NTFS & - & \textit{New Technology File System}\\ 
 RK & - & \textit{Read Key}\\ 
 RPC & - & \textit{Remote Procedure Call}\\ 
 SAD & - & Sistemas de Arquivos Distribuídos\\ 
 SA & - & Sistemas de Arquivos\\ 
 SD & - & Sistema Distribuído\\ 
 SHA & - & \textit{Secure Hash Algorithm}\\ 
 UML & - & \textit{Unified Modeling Language}\\ 
 VK & - & \textit{Verify Key}\\ 
 WK & - & \textit{Write Key}\\ 
 XML & - & \textit{eXtensible Markup Language} \\ 
\end{tabular}
    
\end{siglas}
% ---

\begin{comment}
% ---
% inserir lista de símbolos
% ---
\begin{simbolos}


\end{simbolos}
% ---
\end{comment}


% ---
% inserir o sumario
% ---
\pdfbookmark[0]{\contentsname}{toc}
\tableofcontents*
\cleardoublepage
% ---



% ----------------------------------------------------------
% ELEMENTOS TEXTUAIS
% ----------------------------------------------------------
\textual
% Revisado por Gabriel Saraiva - 14:02

\chapter{Introdução}
\label{cap1}

    A rápida evolução da computação, devido a concorrência entre os fabricantes e sua demanda perene, possibilitou o barateamento e o aumento na capacidade dos sistemas computacionais. Resultado de grandes avanços tecnológicos, dos microprocessadores aliado aos sistemas de conexão e redes, a computação se tornou onipresente em muitos campos de nossa sociedade, como destacado por ~\cite{tanenbaum}.

    A evolução desses sistemas proporcionou a solução de problemas maiores e mais complexos, que comumente não são possíveis de serem solucionados com dispositivos isolados. Para a solução desses problemas são utilizados sistemas distribuídos, onde sistemas computacionais diversos cooperam, se comunicando via redes de computadores e internet para atingir um objetivo comum ~\cite{coulouris}.

    Dentro da computação distribuída, como é necessário armazenar esses dados e fazer o compartilhamento deles, existem os sistemas de arquivos distribuídos, que são responsáveis por fornecer acesso aos dados de forma constante e confiável. Nesse trabalho será tratado o FlexA (\textit{Flexible and Adaptable Distributed File System}), desenvolvido no GSPD (Grupo de Sistemas Paralelos e Distribuídos) da UNESP.


\section{Motivação}

    Dentre os fatores que motivam esse trabalho estão a necessidade de tornar o FlexA um sistema mais fácil de ser expandido no futuro, através de  interfaces que tornem o projeto mais aberto e flexível com relação ao desenvolvimento e arquitetura. Também, aproveitando a grande popularização de dispositivos de computação móvel como, por exemplo, \textit{tablets} e \textit{smartphones}, que com a presença e uso constante desses dispositivos o FlexA poderia ser utilizado pelo usuário de forma prática e rápida.

\section{Objetivos}

    Os objetivos gerais desse trabalho são adaptar as interfaces de comunicação do FlexA, de modo a remodelar interfaces de comunicação entre cliente e servidor, e implementar uma versão do módulo cliente do sistema para dispositivos móveis que utilizem Android.

    Os objetivos específicos são a implementação das operações desconectadas que são responsáveis por gerenciar o FlexA enquanto estiver desconectado e sincronizá-lo ao conectar novamente com os servidores; implementação de um módulo de controle de falhas de conexão, e mecanismos mínimos de segurança dos dados e da comunicação com os servidores, utilização de criptografia e chaves de identificação. Como esses dispositivos possuem uma capacidade de armazenamento menor, existe a necessidade de um melhor gerenciamento dos arquivos residentes na memória cache do sistema de arquivo distribuído (SAD).

\section{Organização do texto}

    No trabalho estão, além dessa introdução, no capítulo \ref{cap2} a fundamentação teórica sobre a qual o trabalho foi desenvolvido, com o estudo de sistemas distribuídos, sistemas de arquivos distribuídos, chamada de procedimentos remotos e também sobre Android e dispositivos móveis. No capítulo \ref{cap3} é descrito o desenvolvimento do projeto e são mostrados os diversos mecanismos utilizados e implementados. O capítulo \ref{cap4} apresenta os cenários de testes utilizados, e os testes realizados para validação do projeto. Por fim, no capítulo \ref{cap5} são feitos comentários sobre a execução do projeto desenvolvido e dos testes realizados, e são apresentadas as dificuldades e novos caminhos para projetos futuros.
    
\chapter{Revisão Bibliográfica}

%%%introdução do capítulo
Neste capítulo são apresentados os conceitos sobre Sistemas de Arquivos Distribuídos (com foco na comunicação entre cliente e servidor), e principalmente a dispositivos móveis, apresentando suas restrições.

%\begin{comment}
%%%%%%%%%%%%%%%%%%%%%%%%% Seção - Computação em nuvem %%%%%%%%%%%%%%%%%%%%%%%%%%%%

\section{Sistemas Distribuídos}

    Segundo ~\cite{tanenbaum}, um Sistema Distribuído (SD) é um conjunto de computadores independentes que se apresenta aos usuários como um sistema único e coerente ocultado detalhes da implementação que não são úteis aos usuários. Ainda sim um sistema distribuído pode ser definido de acordo com ~\cite{coulouris} como um sistema onde os componentes computacionais, tanto os de \textit{hardware} como os de \textit{software}, se comunicam e coordenam suas ações através de mensagens transmitidas por uma ou mais redes de computadores, independente da distância entre seus componentes. Essas características, agregam diversos desafios na implementação de sistemas distribuídos. Esses desafios são:
    
    \subsection{Heterogeneidade}
        Como esses sistemas muitas vezes são compostos por \textit{hardwares} e arquiteturas diferentes, diversos sistemas operacionais e \textit{softwares} desenvolvido por diferentes equipes e muitas vezes com configurações distintas
        entre outros fatores. Dessa forma é de grande importância que o sistema como um todo seja baseado em protocolos compatíveis que possam gerenciar essas diferenças e fornecer ao usuário um ambiente único e coeso.
    
    \subsection{Abertura}
        Um sistema é dito aberto, quando suas interfaces de comunicação ou de desenvolvimento são acessíveis e programas podem ser feitos ou alterados por terceiros para que sejam compatível com esse sistema. Os grandes desafios para criar sistemas abertos são: fornecer principais interfaces publicamente, mecanismos de comunicação uniforme para acesso a recursos compartilhados e por fim, realizar os testes necessários para que esse sistema se comporte como esperado e seja compatível com o padrão publicado.
    
    \subsection{Segurança}
        Um dos grandes desafios de sistemas distribuídos é manter a segurança do próprio sistema de modo a manter a disponibilidade do serviço a seus clientes, confidencialidade dos dados que ele armazena ou processa e não menos importante a integridade do sistema e dos dados que estão no SD. Esse tópico é de grande importância na maioria dos SD disponíveis pois afeta principalmente a disponibilidade do serviço aos usuários e na confiança que o usuário tem sobre o serviço ~\cite{coulouris}.
    
    \subsection{Escalabilidade}
        Um sistema é dito escalável se seu desempenho cresce proporcionalmente aos recursos que são inserido no SD, sem perda de qualidade ou desempenho do serviço. Essa característica é muito importante para que o sistema continue operando em cenários que exijam mais recursos do que o cenário inicial para o qual o sistema implantado. Embora pareça um desafio fácil de solucionar, como mostrado em ~\cite{coulouris}, a solução desses desafios não são sempre triviais, como por exemplo gerenciar a perda de desempenho do sistema ao aumentar o número de servidores e clientes em uma rede, ou manter a latência dentro de um patamar aceitável com o crescimento do número de servidores e a distância entre eles.
    
    \subsection{Tratamento de falhas}

        Como o sistema é distribuído, falhas tendem a não comprometer a totalidade do sistema, afetando apenas alguns pontos  por vez enquanto outros continuam operando. Mas isso torna essas falhas complexas de serem tratadas para que o sistema volte a operar como esperado ~\cite{coulouris}. Para tratar essas falhas são utilizadas as seguintes técnicas:
        \begin{enumerate}
    
            \item Detecção de falha: Erros de transmissão por exemplo podem ser detectados e facilmente corrigidos através de mecanismos próprios e algoritmos bem conhecidos como por exemplo o os algoritmos de \textit{hash} MD5 e o SHA;
            
            \item Mascaramento de falha: Essa técnica consiste não em solucionar o problema mas aumentar a resistência do sistema as falhas, como por exemplo replicar informações servidores e retransmitir mensagens danificadas;
            
            \item Tolerância a falhas: Nem sempre é possível construir sistemas a prova de falhas, sendo muito mais simples e barato projetá-los para tolerar algumas falhas até certo ponto e depois disso informar o usuário sobre a falha ocorrida. Um exemplo clássico citado em ~\cite{coulouris} é o do navegador web que após um certo número de tentativas sem sucesso de carregar uma página \textit{web}, desiste de tentar carregar a página e informa o erro ao usuário;
            
            \item Redundância: Outra forma de tornar o sistema mais tolerante a falhas é acrescentar redundância ao sistema, para que caso uma parte falhe, outras partes detectem essa falha e assumam as tarefas do módulo que falhou sem prejudicar o funcionamento do sistema como um todo. Exemplos são servidores replicados e múltiplas rotas para um destino ~\cite{coulouris};
            
            \item Recuperação de falhas: Mesmo que não seja possível evitar falhas, é importante que o sistema seja capaz de se recuperar delas. Um exemplo disso são sistemas de \textit{Journaling} no sistemas de arquivos locais e de servidores que fazem sincronismo com seus pares após alguma falha que cause a falta de sincronismo do sistema ~\cite{coulouris};
            
        \end{enumerate}
    
    
    \subsection{Transparência}
    
        Dentre diversos recursos para fornecer um sistema coerente ao usuário e esconder dele a complexidade do sistema, estão os conceitos de transparência. Como apresentado em ~\cite{tanenbaum}, é possível listar os conceitos de transparência que o projeto atual utiliza de forma mais intensiva, como por exemplo:
    
        \begin{itemize}
            \item Transparência de acesso: são escondidos do usuários toda a complexidade no acesso dos dados;
            
            \item Transparência de localização: oculta do usuário a localização dos recursos distribuídos;
            
            \item Transparência de migração: é responsável por omitir do usuário detalhes referente a migração dos dados para outras localidades físicas e lógicas;
            
            \item Transparência de replicação: ocultam do usuário os mecanismos de replicação dos dados;
            
            \item Transparência de falhas: esconde do usuário e tenta solucionar os problemas referentes as falhas que ocorrem no sistema, de modo a evitar com que o usuário seja prejudicado quando uma falha acontece no sistema, como visto anteriormente, detectando, mascarando, tolerando, usando módulos redundantes ou até mesmo recuperando o sistema quando possivel.
        \end{itemize}
    
        Embora todas essas camadas de transparência sejam desejadas nem sempre é possível fornecer transparência total ao usuário por questões práticas, como usabilidade, complexidade na implementação e a degradação do desemepnho que elas geram.


\section {Sistema de Arquivos}
    Sistemas de arquivos foram criados para oferecer ao programador uma interface padronizada de acesso aos arquivos, independente de como e onde estão armazenados ~\cite{coulouris} dentro dos dispositivos locais. 
    Além dos dados (arquivos propriamente ditos), sistemas de arquivos devem armazenar também os metadados, que são informações sobre os dados (arquivos). Esses metadados comumente são os seguintes ~\cite{coulouris} (adaptado):
    
    \begin{itemize}
        \item Nome do arquivo;
        \item Dono do arquivo;
        \item Grupo do arquivo;
        \item Horário de criação;
        \item Horário de acesso;
        \item Horário de modificação;
        \item Horário de alteração de atributo;
        \item Contagem de refências;
        \item Tipo do arquivo (diretório, arquivo, link);
        \item Lista de permissões de acesso.
    \end{itemize}
    
    Além de gerenciar como os arquivos são armazenados e acessados, cabe ao sistema de arquivos a responsabilidade de fazer o controle de acesso dos arquivos que ele armazena, restringindo o acesso conforme as regras definidas pelos usuários ou sistema. Essas regras normalmente podem ser expressas pelos acessos de leitura, escrita e execução dos arquivos.
    
    Em um sistema de arquivos convencional todo acesso aos arquivos é controlado pelo \textit{kernel} do sistema operacional. O tratamento do acesso é feito pelo próprio núcleo do sistema. Em sistemas distribuídos esse tratamento é feito de forma diferente. Isso será abordado na sessão de Sistemas de Arquivos Distribuídos.
    
    Exemplos de sistemas de arquivos são: EXT2, EXT3, EXT4, XFS, ReiserFS, JFS para sistemas Unix-like e não menos importantes os FAT16, FAT32, FAT32EX, e NTFS para sistemas da Microsoft.


\section{Sistemas de Arquivos Distribuídos}

	 Sistemas de Arquivos Distribuídos (SAD), são sistemas que oferecem compartilhamento desses arquivos através de redes, fornecendo desempenho, integridade, confidencialidade e disponibilidade equivalentes ou superiores aos sistemas de arquivos tradicionais ~\cite{coulouris}, permitindo que programas armazenem e acesses os arquivos remotos como se fossem locais, fornecendo aos usuários liberdade para acessar seus arquivos independente do computador que utilizem, desde que dentro de uma rede local ou que tenha acesso ao sistema de arquivos distribuído utilizado.
	 
	 Um SAD básico pode ser descrito como é encontrado em ~\cite{coulouris}, em que o objetivo é simplesmente simular o funcionamento de um sistema de arquivos tradicional para que programas clientes executem em computadores remotos. Esses não fazem o uso de sistema de réplicas, nem suportam sistemas multimídia que exigem grandes fluxos de dados e a capacidade de fazer a entrega desses dados com uma temporização bem limitada.
	 Analisado por essa perspectiva que é apresentada em ~\cite{coulouris} o sistema de arquivos distribuído que é alvo desse trabalho pode ser considerado um SAD não básico, pois implementa vários recursos não encontrados em sistemas básicos. Esse sistema será apresentado na próxima sessão.
	 
	 Assim, para projetar um sistema de arquivos distribuído de alto desempenho capaz de armazenar grandes quantidades de dados e que suporte escrita e leitura em larga escala é necessário resolver diversos problemas complexos, como balanceamento de carga, confiabilidade, disponibilidade, segurança, e concorrência de acesso aos dados ~\cite{coulouris}.
	 A seguir são elencados alguns sistemas de arquivos distribuídos que são referências clássicas e foram a base do desenvolvimento desse trabalho.
	
	 
	\subsection{\textit{Network File System} (NFS)}
	
	    O \textit{Network File System} (NFS) foi projetado pela \textit{Sun Microsystems} para funcionar em suas estações de trabalho. Atualmente é compatível com quase todos os sistemas operacionais de grande uso. Foi lançado como o primeiro SAD destinado a ser um produto final e ter suas interfaces e protocolo abertos ao público. Sua grande compatibilidade somada a premissa de fornecer acesso transparente aos arquivos remotos através do uso das chamadas ao \textit{Virtual File System} (VSF) e do uso de \textit{Remote Procedure Call} RPC foram os grandes fatores que contribuíram para seu sucesso e sua popularização ~\cite{coulouris}. Sua arquitetura é apresentada na Figura \ref{fig:arquiteturaNFS} .
	
    	\begin{figure}[h]
            \includegraphics[width=12cm]{arquiteturaNFS.png}
            \caption{Arquitetura do NFS ~\cite{coulouris}}
            \label{fig:arquiteturaNFS}
        \end{figure}
    
    Seu projeto independente do estado do sistema (\textit{stateless}) basicamente inviabiliza a utilização de mecanismos para recuperação de falhas ~\cite{coulouris}.
    
    O uso de cache no servidor tenta antecipar o que o cliente irá solicitar e já manter esses dados na memória para quando o cliente solicita-los possam ser entregues sem o gargalo de acesso ao disco. O uso de cache na escrita se baseia em manter o arquivo na memória por um tempo sem que o cliente o altere e só então gravar o arquivo no disco. Ainda sim existe a operação \textit{sync} que grava as informações pendentes no disco a cada 30 segundos.
    Quanto a cache no cliente é feita através da datas de modificação dos arquivos. Caso haja discrepâncias entre cliente e servidor, o cliente é atualizado ~\cite{coulouris}.
    
    Para questões de segurança o NFS utiliza dois mecanismos externos. Para a transmissão dos dados, autenticação e comunicação do cliente com o servidor é utilizado o protocolo Kerberos v5 junto com RPCSEC\_GSS. Para proteger os dados nos clientes é utilizado o mecanismos de lista de controle de acesso (\textit{Access Control List} - ACL) que é responsável por definir as permissões para acesso aos arquivos. ~\cite{tanenbaum}.
    
    Quanto a desempenho, o NFS fornece uma solução satisfatória até mesmo para sistemas com grande demanda, ao utilizar diversos servidores. Mas essa solução é limitada pelo fato do sistema não fornecer suporte a replicação de arquivos para leitura e escrita (apenas a replicação para leitura é suportada). Ainda sim, mesmo que bastante difundido e utilizado o NFS carece de mecanismos de migração de dados e réplicas que estão presente em outros SAD mais avançados ~\cite{coulouris}.
    
    \subsection{\textit{Andrew File System} (AFS)}
    
    Da mesma forma que o NFS visto na sessão anterior, o \textit{Andrew File System} (AFS) além de ser compatível com o NFS também fornece acesso de forma transparente aos arquivos, embora utilize uma abordagem um pouco diferente para atingir esse objetivo. Tendo como objetivo principal a escalabilidade do sistema seus mecanismos de cache no cliente são bem agressivos e nas versões mais atuais o sistema passou a trabalhar armazenando o estado dos clientes com o uso de de \textit{callbacks} para diminuir a comunicação entre cliente e servidor e manter baixo o uso dos processadores nos servidores como apresentado por ~\cite{coulouris}.
    
    O AFS tem seu projeto baseado na suposição sobre o tamanho médio e máximo dos arquivos em discos nos sistemas \textit{UNIX} de ambientes acadêmicos e outros. De acordo com ~\cite{coulouris} as observações mais importantes sobre o estudo que direcionou a implementação do AFS são:
    \begin{itemize}
        \item Os arquivos costumam ser pequenos (normalmente com menos de 10KiB).
        \item Operações de leitura são muito mais frequentes que as de escrita (6 vezes mais recorrentes).
        \item O acesso aos arquivos é feito de forma sequencial na maioria das vezes.
        \item A maioria dos arquivos é lida e escrita por apenas um usuário, e mesmo quando compartilhado a maioria das vezes apenas um único usuário é responsável por escrever no arquivo.
        \item Os arquivos tendem a ser referenciados repetidas vezes em momentos próximos.
    \end{itemize}
    
    Com base nessas premissas o AFS opera da seguinte forma:
    \begin{itemize}
        \item Servir arquivos inteiros.
        \item Cache de arquivos inteiros nos clientes.
        \item Uso de cache persistente nos clientes (armazenados em disco).
        \item Uso de um kernel modificado nos clientes para interceptar as chamadas de acesso aos arquivos.
    \end{itemize}
    
    Com base nos estudos feitos e nas características citadas acima é fácil perceber que existe um grupo de arquivos que é totalmente fora desse escopo: arquivos de bancos de dados. Dessa forma o projeto do AFS declaram explicitamente que o objetivo do sistema não é atender a essa categoria de arquivos.
    
    A arquitetura do AFS utiliza servidores e clientes. O \textit{software} utilizado nos servidores é chamado de Vice, enquanto o que executa nas estações clientes é o Venus, como pode ser visto na Figura \ref{fig:arquiteturaAFS} .
    
    \begin{figure}[h]
        \centering
        \includegraphics[width=12cm]{arquiteturaAFS.png}
        \caption{Arquitetura do AFS ~\cite{coulouris}}
        \label{fig:arquiteturaAFS}
    \end{figure}
    
    O gerenciamento de cache no cliente é tratado pelo processo Venus, que ao receber uma operação open verifica se o arquivo solicitado já existe na cache e se existe alguma versão mais nova desse arquivo nos servidores. Caso o arquivo da cache esteja defasado a nova versão é solicitada aos servidores e é fornecida ao programa que fez solicitou o arquivo. Quando o arquivo é fechado através da operação \textit{close} o processo Venus (cliente) verifica se o arquivo foi alterado e caso afirmativo envia a nova versão do arquivo ao servidor. É importante ainda ressaltar que o AFS não dispõe de nenhum mecanismo para tratar atualização concorrente de um mesmo arquivo, de forma que se vários clientes enviarem versões diferentes do mesmo arquivo ao servidor, todas exceto a última a ser processada pelo servidor (Vice) serão perdidas sem nenhum aviso ou erro, restando apenas a última requisição processada pelo servidor, que será a versão do arquivo que será salva e estará disponivel ~\cite{coulouris}.
    
    Quanto aos metadados vale citar que os banco de dados ficam replicados integralmente em todos os servidores (Vice) ~\cite{coulouris}, de forma que cada servidor sabe exatamente onde se encontram os arquivos.
    
    Semelhante ao NFS, o AFS só suporta réplica de arquivos para leitura, direcionando todas as escritas nesse arquivo para apenas um servidor, que então faz o sincronismo posteriormente e tem que ser feita de forma explicita ~\cite{coulouris}.
    
    As caracteristicas citadas acima fornecem ao AFS um bom desempenho para cargas relativamente grandes sem sobrecarregar demasiadamente os servidores, o que fornece grande taxa de escalabilidade, sendo em alguns casos até 60\% mais eficiente no uso de CPU que o NFS com a mesma carga de trabalho ~\cite{coulouris}.
    
    \subsection{\textit{Tahoe-LAFS- Least Authority File System}}
    
    Escrever.
    
    \subsection{CODA \textit{Distributed File System}}
    
    Escrever.
    
    \subsection{\textit{Hadoop File System}}
    
    Escrever.
    
    \subsection{\textit{Flexible and Adaptable distributed file system} (FlexA)}
	 
	 O \textit{Flexible and Adaptable distributed file system} (FlexA) ~\cite{silas}, desenvolvido pelo Grupo de Sistemas Paralelos e Distribuídos (GSPD) que é o objeto de estudo desse trabalho, tem foco na utilização de recursos computacionais das estações clientes para diminuir a carga de processamento dos servidores, além de fornecer um sistema de segurança descentralizado e o uso de mecanismo de tolerância a falhas, aproveitando características de diversos SADs, focando em fornecer um SAD com grande escalabilidade ~\cite{silas}.
	
	 O FlexA mudou muito desde sua versão inicial, desenvolvida em 2012 ~\cite{mario}. A versão que será tratada nesse texto é a última versão que ainda está em desenvolvimento, baseada na versão original. Essa versão é o resultado da cooperação de diversos integrantes do GSPD ~\cite{mario}, durante a evolução do sistema nesses dois anos desde sua versão inicial. A arquitetura do sistema original pode ser vista na Figura \ref{fig:arquiteturaFlexaOriginal} .
	 
	 \begin{figure}
	 \centering
	 \includegraphics[width=14cm]{arquiteturaFlexAOriginal.png}
	 \caption{Arquitetura original do FlexA ~\cite{silas}.}
	 \label{fig:arquiteturaFlexaOriginal}
	 \end{figure}
	 
	 A arquitetura atual do projeto, se baseia muito no que foi proposto por ~\cite{silas} é apresentada na Figura \ref{fig:arquiteturaFlexa} .
	 
	 \begin{figure}
	 \centering
	 \includegraphics[width=14cm]{arquiteturaFlexA.png}
	 \caption{Arquitetura atual do FlexA ~\cite{mario}.}
	 \label{fig:arquiteturaFlexa}
	 \end{figure}
	 
	 Para atingir os objetivos descritos acima o FlexA possui diversos mecanismos que serão elencados nas sessões a seguir.
	 
	 \subsection{Segurança}
	    A segurança do sistema é baseada no controle de acesso aos arquivos e também no controle de escrita sobre o mesmo ~\cite{silas}.
	 
    	 \begin{itemize}
    	 
    	    \item Controle de Acesso: é implementado utilizando um trio de chaves para cada arquivo. As chaves que compõe esse trio são:
        	    \begin{itemize}
        	        \item \textit{Verify Key} (VK): fornece acesso ao arquivo, servindo como um identificado único do arquivo no sistema.
        	        \item \textit{Read Key} (RK): é a chave utilizada para cifrar o arquivo. Sem essa chave, mesmo que seja conhecida a VK não é possivel ler o conteúdo do arquivo.
        	        
        	        \item \textit{Write Key} (WK): fornece acesso de escrita no arquivo dentro dos servidores.
        	    \end{itemize}
        	    
            Das três chaves citadas acima a única que não é enviada aos servidores é a chave usada na criptografia do arquivo, a \textit{Read Key}. Dessa forma mesmo que o servidor seja comprometido não é possível obter o conteúdo original dos arquivos que ele armazena ~\cite{mario}.
        	    
        	    
        	    Essas chaves são criadas com base em uma chave privada RSA ~\cite{shamirRSA} especificada pelo usuário, junto com um valor único para cada arquivo chamado de \textit{salt} que é fornecido pelo servidor. Uma vez com a RSA e o \textit{\textit{salt}} o processo feito para obtenção do trio de chaves é o apresentado na Figura \ref{fig:chavesFlexa}.
        	    
        	    \begin{figure}
        	    \centering
        	    \includegraphics[width=10cm]{chaves.png}
        	    \caption{Diagrama sobre a criação das chaves de acesso e criptografia do FlexA ~\cite{mario}.}
        	    \label{fig:chavesFlexa}
        	    \end{figure}
        	    
        	    Como mostrado na Figura \ref{fig:chavesFlexa} a chave de identificação do arquivo (VK) é gerada utilizando a o \textit{hash} SHA512 da RSA do usuário concatenada com o do \textit{salt} retornado pelo servidor.
        	    Após isso a chave de criptografia (RK) é gerada utilizando a o hash SHA386 da RSA do usuário concatenada com o do identificador do arquivo (\textit{Verify Key});
        	    Por fim é gerado a chave de escrita (WK), utilizando a RSA do usuário concatenada com a chave de leitura.
        	    
        	    Todo o processo criptográfico e de geração de chaves é feito de forma transparente, sem que o usuário tenha que fazer isso de forma manual, notando apenas o tempo que é necessário para fazer a criptografia / descriptografia do arquivo.
    	    
        \item Integridade: O mecanismo de integridade do arquivo é implementado utilizando a chave WK mostrada anteriormente. A permissão de escrita (chave WK) é solicitada sempre que um cliente deseja fazer a operação de escrita (atualização) de um arquivo nos servidores. Dessa forma o cliente deve fornecer a WK correspondente ao arquivo que deseja atualizar, então o servidor comprara a WK fornecida com a armazenada e caso sejam idênticas o servidor faz a substituição do arquivo antigo pelo novo.
        
        Mecanismos para evitar ataques de repetição e outros semelhantes ainda precisam ser implementadas no FlexA, pois a transmissão das chaves de acesso e escrita podem ser facilmente capturadas e reutilizadas com a implementação atual.
        
    \end{itemize}
    
    \subsection{Tolerância a falhas e Adaptabilidade}
        
        Os mecanismos utilizados pelo FlexA para executar esse objetivo são o uso de réplicas das porções dos arquivos, que são enviados para diversos servidores e lá são replicados com o grupo de servidores secundários. Também é utilizada a possibilidade de um servidor secundário assumir além da sua função a de um servidor primário caso seja necessário por falha de algum primário ou sobrecarga do grupo de servidores primários ~\cite{mario}.
        
        Sempre que um servidor primário ou secundário falha, é executado um processo que faz o sincronismo do servidor com os outros servidores ativos para que esse assuma o estado atual do sistema e volte o mais breve possível a colaborar com o atendimento dos clientes ~\cite{silas}.
        
    
    \subsection{Escalabilidade}
    
        Os principais fatores que auxiliam na escalabilidade do sistema são:
        
        \begin{itemize}
            \item Criptografia e descriptografia dos arquivos nos clientes: Evitam o consumo de UCP nos servidores, deixando-os mais disponíveis para atender a novas requisições.
            \item Divisão dos arquivos em diversas porções nos clientes: Diminuem o uso de armazenamento e uso dos discos dos servidores.
            \item União das diversas porções que compõe um arquivo nos clientes: Também influenciam no uso de disco dos servidores, liberando os discos para atenderem outras requisições mais rapidamente.
            \item Leitura e escrita das porções dos arquivos de diferentes servidores, evitam sobrecarregar o uso de banda de apenas um servidor e aumentam o uso da banda no cliente.
            \item Utilização agressiva de cache nos clientes: Essa técnica evita acessos recorrentes aos servidores, principalmente para arquivos que são pouco atualizado ou são utilizados por apenas um usuário.
            \item Uso de sistemas de replicação para servidores secundários: Auxiliam nas operações de leitura dos arquivos pelos clientes.
        \end{itemize}
        
        
    \subsection{Flexibilidade}
    
        Como boa parte da carga de processamento e uso do disco no FlexA é transferida aos clientes, o sistema pode operar com \textit{hardware} de baixo custo sem grandes problemas. Isso ainda fornece ao FlexA a possibilidade de que caso seja necessário clientes com mais recursos disponíveis podem passar a fazer parte do grupo de servidores auxiliando no atendimento a requisições de outros clientes. Essa característica faz com que o sistema aproveite muito recurso que estaria ocioso em sua rede, principalmente em momentos de uso intensivo dos servidores ~\cite{silas}.
        
        Além disso o sistema é projetado para que seja possível fazer a troca dos mecanismos de segurança, regras de gerenciamento da cache nos clientes,  métricas de divisão das porções e diversos outros elementos do sistema de forma simples, principalmente devido a escolha do Python como linguagem de programação, facilita o acesso ao código fonte do FlexA ~\cite{silas}.
        
    \subsection{Abertura}
    
    A abertura do FlexA é dada basicamente pelo fato de o sistema ter o código fonte aberto e disponível ~\cite{silas}. Ainda que carente de uma documentação concisa e robusta, é possível dar continuidade ao projeto.


\section{Chamada de Procedimentos Remotos (RPC)}

    Para realizar comunicação entre cliente e servidores é possível utilizar diversas técnicas diferentes. Nesse trabalho a Chamada de Procedimento Remoto ou \textit{Remote Procedure Call} (RPC) é uma ferramenta muito utilizada pois é a partir dela que a maior parte das comunicações entre cliente e servidor acontecem. A escolha desse paradigma foi feita pela simplicidade do projeto e desenvolvimento. 
    
    O RPC utiliza um paradigma de comunicação de alto nível, ocultando do desenvolvedor quase todo o processo de estabelecimento de conexão, transmissão dos dados, conversão dos dados e bloqueio do cliente ~\cite{rpc} ao realizar requisições.
    
    A descrição básica da comunicação via RPC é feita em ~\cite{rpc}. Para a utilização do RPC na implementação de um serviço é necessário um processo chamado de servidor RPC (Servidor) que possui as funções disponíveis que serão solicitadas por um cliente através de uma requisição RPC. Assim cabe ao Servidor registrar todas as funções que o cliente pode requisitar. Após registrar as operações o Servidor inicia a escuta por requisições.
    
    Em outro \textit{host}, um processo chamado Cliente faz uma Requisição RPC ao Servidor. O Servidor executa a função solicitada, com os parâmetros enviados e retorna o resultado ao cliente.
    
    De acordo com ~\cite{rpc} o uso de RPC reduz em média 50\% da complexidade no desenvolvimento da comunicação entre cliente e servidor.
    
    \subsection{XML-RPC}
    
    No mercado existem diversas soluções para o uso de RPC. O XML-RPC é uma implementação de RPC que utiliza TPC/IP junto com HTTP e XML para a troca de mensagens entre servidor e cliente ~\cite{xmlrpc}. O uso de um padrão como o XML-RPC é também justificado pela representação dos dados que é mantida através de diversas arquiteturas ~\cite{xmlrpcMessage}
    
    A versão atual do FlexA utiliza o XML-RPC para todas as comunicações entre cliente/servidor exceto para a transmissão de arquivos ~\cite{mario}. A transmissão dos arquivos é feita via \textit{socket} para evitar o processo de \textit{marshalling} dos arquivos que tende a ser demasiadamente custoso em questões de uso de UCP e demorado, dessa forma reduzindo o tempo de entrega dos arquivos ao cliente.
    
    Um diagrama bem simples sobre o funcionamento desse paradigma descrito acima é exibido na Figura \ref{fig:xmlrpc} na página .
    
    \begin{figure}[ht]
    \centering
    \includegraphics[width=10cm]{xmlrpc.png}
    \caption{Diagrama que exibe a comunicação via XML-RPC. ~\cite{xmlrpc} (Adaptado)}
    \label{fig:xmlrpc}
    \end{figure}
    
    Como pode ser visto na \ref{fig:xmlrpc}, o cliente faz a requisição de uma função remota, passando os parâmetros. Toda a comunicação é feita com o uso de XML. O servidor recebe a requisição, processa e retorna o resultado também em XML. Um outro exemplo mostra como é feita a comunicação de uma requisição com XML-RPC como pode ser visto na Figura ~\ref{fig:xmlrpcMessage} . O processo de resposta da requisição segue o mesmo principio.
    
    \begin{figure}[ht]
    \centering
    \includegraphics[width=10cm]{xmlrpcMessage.png}
    \caption{Exemplo de mensagem XML utilizada na chamada de procedimentos remotos com XML-RPC}
    \label{fig:xmlrpcMessage}
    \end{figure}


\section{Dispositivos Móveis}

    Quanto a classificação de dispositivos móveis, é muito difícil fornecer uma definição formal que seja adequada. Assim essa definição pode ser feita em duas partes, como feita em ~\cite{mobileDevices}.
    
    \subsection{Dispositivos Móveis}
    De acordo com ~\cite{mobileDevices}, dispositivos móveis são dispositivos portáveis como    \textit{laptops}, \textit{Personal Digital Assistants} (PDA), \textit{tablets}, \textit{smart phones}, \textit{handhelds}, \textit{MP3 Players}, consoles de jogos portáteis entre outros. Variando de dispositivos com pouquíssima autonomia, baixa ou nula conectividade e pouca capacidade computacional até dispositivos de ultima geração com autonomia relativamente alta (algumas dezenas de horas de uso constante), potencia computacional muitas vezes superior a computadores de mesa e \textit{notebooks} e grande conectividade utilizando diversas tecnologias diferentes. 
    
    
    \subsection{Computação Móvel}
    Dentro do escopo desse trabalho mais importante que a definição de dispositivos móveis é a definição  computação móvel, que é apresentada por ~\cite{mobileDevices} como sendo um conjunto de dispositivos móveis que fornecem ao usuário um sistema computacional que pode operar em dois modos: conectado e desconectado. Quando desconectado de redes de dados esses dispositivos devem funcionar de forma dessincronizada com sua fonte de dados, e quando conectado novamente a redes que fornecem acesso a transmissão de dados deve fazer o sincronismo dos dados através de operações de \textit{upload/download}.
    
    
\section{Android}

Na maioria das vezes é necessário um sistema operacional (SO) que gerencie um dispositivo móvel, esse sistema operacional pode ser o Android ~\cite{android} ou outro sistema operacional comum ou preferencialmente próprio para dispositivos desse tipo. Existem atualmente no mercado dezenas de SOs diferentes para dispositivos móveis, mas os que mais se destacam frente aos usuários e grandes fabricantes são Android (\textit{Open Handheld Alliance}) e iOS (Apple) ~\cite{mobileDevicesMarketShare}.

Mais do que um sisples sistema operacional o Android é composto por diversas camadas de software (~\textit{Android Software Stack}) que fornecem ao sistema suporte a uma grande variedade de dispositivos e flexibilidade no seu uso ~\cite{android}. Essas camadas são exibidas na Figura \ref{fig:androidStack} na página.

\begin{figure}[h]
\centering
\includegraphics[width=14cm]{androidStack.png}
\caption{Camadas de software que compõe o sistema Android ~\cite{android}.}
\label{fig:androidStack}
\end{figure}


% Ideia desse capitulo: Apresentar o desenvolvimento do projeto-
% 
% Deficiências do projeto atual         OK
% Propostas de melhorias                OK
% Documentação do Projeto               OK
% Adaptação do servidor python          GOGOGO
% Implementação do cliente Android      GOGOGO

\chapter{Descrição e desenvolvimento do projeto}


    Neste capítulo são apresentados o desenvolvimento da proposta do projeto e também sua implementação no FlexA. Inicialmente serão abordados as deficiências encontradas no projeto atual e a motivação para corrigi-las. Então será apresentada a documentação desenvolvida para que o projeto possa ser viável a longo prazo, e em outra sessão será detalhado as adaptações realizadas no módulo servidor para que esse passasse a ser compatível com a nova especificação. Por fim, com o novo servidor pronto, será apresentado a implementação do cliente para Android.

    \section{Deficiências da versão atual do FlexA}
    
    
        A versão atual do FlexA, ainda em um estágio inicial de desenvolvimento, carece de um modelo de programação bem definido, documentação formal e não menos importante de um bom protocolo de comunicação robusto.
        Os principais pontos que serão abordados nesse projeto para corrigir essas deficiências serão:
        
        \begin{itemize}
            \item Documentação documentação inicial do projeto, como um diagrama de classes, casos de uso e também um diagrama com os módulos do sistema e como eles se integram.
            \item Elaboração de um protocolo de comunicação bem definido.
            \item Adaptação incremental do módulo servidor para que ele passe a ser compatível com a documentação gerada.
        \end{itemize}
        
        
        A motivação para essas melhorias é que elas fornecerão uma base sólida para que no futuro possam ser incorporados no projeto atual os resultados de trabalhos já realizados em versões anteriores do FlexA, que são listados pelos títulos:
        
        \begin{itemize}
            \item Detecção de Falhas de Comunicação e Balanceamento de Carga no FlexA ~\cite{danilo};
            \item Metodologia para Recuperação de Falhas e Garantia de Disponibilidade no FlexA ~\cite{thiago};
            \item Implementação e avaliação de desempenho de algoritmo de criptografia em GPU para o FlexA ~\cite{leandro};
            \item Sincronização, consistência e falhas no FlexA ~\cite{matheus};
            \item Disponibilidade em um sistema de arquivos distribuído flexível e adaptável ~\cite{lucio};
        \end{itemize}
        
       
    Além de tornar mais fácil a incorporação dos resultados obtidos com os trabalhos citados, a nova documentação deve tornar o sistema fácil de se manter e evoluir e permitir que diversas equipes colaborem com o trabalho de forma simultânea.
    
    
    
        
    \section{Documentação do projeto}
        
    Um estudo a fundo do sistema foi realizado, utilizando a documentação que existia e a análise do código-fonte, para entender o real funcionamento e comportamento do sistema e como era definida a estrutura atual do projeto. 
    
    \subsection{Módulos do Sistema}
    
        Para representar o sistema de forma mais abstrata foi criado um diagrama dos módulos do sistema atual, e suas dependências conforme pode ser visto na Figura \ref{fig:pacotesMario}.
        
        \begin{figure}[!ht]
            \centering
            \includegraphics[width=10cm]{pacotesMario.png}
            \caption{Módulos do sistema atual e suas dependências. Criado com base no código fonte do projeto ~\cite{mario}.}
            \label{fig:pacotesMario}
        \end{figure}
        
        Esse diagrama foi gerado a partir do estudo do código-fonte da versão atual do projeto. Apenas por esse diagrama já é possível notar que o módulo Servidor não utiliza os serviços de criptografia, e que apenas o servidor tem acesso aos bancos de dados de metadados, o que ajuda a evidenciar a segurança dos dados do cliente que já chegam aos servidores criptografardes.
    
    \subsection{Diagrama de classes do sistema existente} 
    
        Para que fosse possível analisar a estrutura mais interna dos módulos foi criado o diagrama de classes da \textit{Unified Language Model} (UML)~\cite{umlClasses}. Esse diagrama é apresentado na figura \ref{fig:classesMario}. Embora seja um diagrama de classes, alguns abusos de notação tiveram que ser utilizados devido a lacuna existente entre a representação dos diagramas de Classes da e a linguagem Python em que o FlexA é desenvolvido.
        
        \begin{figure}[!ht]
        \centering
        \includegraphics[width=14cm]{classesMario.png}
        \caption{Classes e pacotes (arquivos) que estruturam a versão atual do FlexA}
        \label{fig:classesMario}
        \end{figure}
             

    
        Analisando o diagrama de classes apresentado na Figura \ref{fig:classesMario} é possível perceber que muitas atividades diferentes e classes distintas pertencem a um mesmo pacote. 
        
        
        Conforme a proposta desse trabalho, desenvolver um cliente para Android do FlexA e melhorar os quesitos de abertura do FlexA, se mostrou necessário realizar alterações no projeto para que fosse possível a execução desses objetivos sem comprometer a flexibilidade do projeto no futuro.
        
        \subsection{Diagrama de Casos de Uso}
        
        Para que fosse possivel elaborar a adaptação do projeto, num primeiro momento foi feito o levantamento dos requisitos do FlexA, junto com os objetivos de melhoria do código existente. Dessa forma foi elaborado o seguinte diagrama UML de casos de uso de acordo com ~\cite{umlCasosDeUso}. O diagrama com os casos de uso é apresentado na Figura \ref{fig:casosDeUsoGabriel}.
        
        \begin{figure}[!ht]
        \centering
        \includegraphics[width=15cm]{casosDeUsoGabriel.png}
        \caption{Diagrama de Casos de uso com os requisitos funcionais do FlexA.}
        \label{fig:casosDeUsoGabriel}
        \end{figure}
        
        
        Dos casos de uso apresentados na figura \ref{fig:casosDeUsoGabriel}, serão implementados e utilizados nesse trabalho apenas os referentes ao módulo cliente devido ao escopo do projeto. Os restantes foram definidos juntos para preparar os trabalhos futuros, uma vez que o momento era oportuno.
        
        
        
        \subsection{Interface de Comunicação Cliente-Servidor}
        
        Já com os objetivos do trabalho definidos, e uma documentação básica criada que permitisse entender o projeto foi então formulada a interface que servirá de alicerce desse trabalho. Responsável por padronizar e fornecer as comunicações entre o cliente e o servidor. Essa interface foi desenvolvida em reunião com os outros desenvolvedores do FlexA para que fossem atendidas todas as necessidades do projeto, e é apresentada na figura \ref{fig:interfaceComunicacao}
       
        \begin{figure}
        \centering
        \includegraphics[width=14cm]{interfaceCom.png}
        \caption{Interface de comunicação dos servidores do FlexA.}
        \label{fig:interfaceComunicacao}
        \end{figure}
        
        Essa interface é de grande importância para o projeto, pois é ela que servirá como referencia para o desenvolvimento do módulo cliente e também do módulo servidor, uma vez que define os métodos que deverão ser invocados durante a comunicação entre os clientes e os servidores, utilizando XML-RPC e independendo da linguagem utilizada.
        
        \subsection{Protocolos de Comunicação Cliente-Servidor}
        
        Com as funcionalidades que são esperadas do sistema e a interface de comunicação definida, foi feita a analise das comunicações entre cliente e servidor. Formalizou-se então os protocolos de comunicação, utilizando a notação para protocolos de comunicação apresentada em ~\cite{ross} que é mostrada na tabela \ref{tab:notacao}.
        
        \begin{table}
        
        \centering
        \begin{tabular}{|l|l|}
        \hline
        
        $C$ & módulo cliente ou usuário \\
        \hline
        $S$ & módulo servidor \\
        \hline
        $privKey_{X}$ & chave privada $X$ \\
        \hline
        $pubKey_{X}$ & chave pública $X$ correspondente a $privKey_{X}$\\
        \hline
        $C \rightarrow S : dado$ & $C$ envia $dado$ para $S$\\
        \hline
        $C \rightarrow S : {dado}_{pubKey_{x}}$ & $C$ envia $dado$ criptografado com a chave pública $X$ para $S$\\
        \hline

        \end{tabular}
        \caption{Notação utilizada para definição de protocolos de comunicação ~\cite{ross}.}
        \label{tab:notacao}
        \end{table}
        
        Formalizadas as definições que serão utilizadas a seguir, são apresentados os protocolos para as comunicações. Esses protocolos foram elaborados baseados no código fonte ~\cite{mario} e na interface de comunicação apresentada, já inserindo as adaptações para fornecer maior modularização, compatibilidade e segurança ao sistema.
        
        \subsubsection{Protocolos de manipulação de arquivos}
        
        A primeira formalização é o protocolo para solicitação dos metadados do usuário, que é feita quando o FlexA é iniciado pela primeira vez ou caso seja necessário obter o \textit{userID} do usuário. Isso é feito com base em sua chave pública previamente cadastrada no sistema por um administrador. O protocolo é apresentado na figura \ref{fig:protMetadadosUsuario}. Esse protocolo é implementado pela função \textit{get\_user\_metadata}, que é apresentada na figura \ref{fig:interfaceComunicacao}.
        
        \begin{figure}[!ht]
        
        \bordaProtocolo{
                $C \rightarrow S: pubKey_{C}$ \\
                $S \rightarrow C: \{userID,homeKey\}_{pubKey_{C}}$
        }
        
        \caption{Protocolo de requisição dos metadados de um usuário com base na sua chave pública.}
        \label{fig:protMetadadosUsuario}
        \end{figure}
        
        Com os metadados do usuário (\textit{userID} e \textit{homeKey} que é o identificador se sua pasta raiz) o usuário pode começar a utilizar o sistema enviando e recebendo arquivos.
        
        Para que o usuário possa enviar um arquivo para o sistema é necessário verificar se existe um arquivo com o mesmo nome no mesmo endereço especificado através do \textit{salt}. Caso o arquivo não exista ainda, é feito o cadastro e solicitado o \textit{salt} referente ao arquivo. Na figura \ref{fig:protGetSalt} é formalizado o protocolo de requisição do \textit{salt} de um arquivo. É importante ressaltar que \textit{fileName} é composto pelo nome completo do arquivo com o diretório em que o arquivo se encontra. Esse endereço é relativo ao diretório mapeado do FlexA para o usuário. Esse protocolo é implementado pela função \textit{get\_salt}, que é apresentada na figura \ref{fig:interfaceComunicacao}.

        \begin{figure}[!ht]
        \bordaProtocolo{
            $C \rightarrow S: fileName,userID$ \\
            $S \rightarrow C: salt$
        }
        \caption{Requisição do \textit{salt} de um arquivo.}
        \label{fig:protGetSalt}
        \end{figure}
        
        Caso o \textit{salt} retornado pelo servidor seja igual $0$ é assumido que esse arquivo ainda não existe no servidor com esse nome, e então o próprio cliente gera um \textit{salt} para esse arquivo.
        
        Já com o \textit{salt}, o módulo cliente gera o trio de chaves para o arquivo (VK, RK e WK), criptografa o arquivo utilizando a RK, faz a divisão do arquivo em $N$ porções de acordo com o tamanho do arquivo, ($N$ e o tamanho do arquivo são configurações definidas pelo usuário) caso necessário e envia cada porção para um servidor. Primeiro o módulo cliente faz o cadastro do arquivo no servidor e então envia as porções. Na figura \ref{fig:protSendFile} é descrito o protocolo o cadastro do arquivo, onde $N$ é o número de porções em que o arquivo foi dividido, $directoryKey$ é o identificador do diretório que o arquivo está e $fileType$ é o tipo do arquivo (diretório ou arquivo comum). Esse protocolo é implementado pela função \textit{negociate\_store\_part}, que é apresentada na figura \ref{fig:interfaceComunicacao}.
        
        \begin{figure}[!ht]
        \bordaProtocolo{
            Para $i$ de $1$ até $N$: \\
            $C \rightarrow S_{i}: userID,fileName,verifyKey,directoryKey,writeKey,salt,partNumber_{i}$ \\
            $S_{i} \rightarrow C: port $
        }
        
        \caption{Cadastro do arquivo nos servidores e negociação da porta de envio do arquivo}
        \label{fig:protSendFile}
        \end{figure}
        
        Após o cadastro do arquivo e a negociação da porta de transmissão do arquivo, é feita a transmissão do arquivo em uma nova conexão com o servidor na porta $port$.
        
        Com a negociação pronta basta enviar para o servidor a porção correspondente. Essa comunicação é definida conforme o protocolo mostrado na figura \ref{fig:protSendFileData}. Esse protocolo é implementado através de sockets, sem o uso do XML-RPC, por questões de desempenho.
        
        \begin{figure}[!ht]
        \bordaProtocolo{
            Para $i$ de $1$ até $N$: \\
            $C \rightarrow S_{i}: filePart_{i}$
        }
        \caption{Transmissão das porções dos arquivos aos servidores, feito via socket.}
        \label{fig:protSendFileData}
        \end{figure}
        
        
        Com os protocolos já definidos é possível enviar um arquivo para o servidor. Mas ainda não é possível recupera-lo. A seguir serão tratados os protocolos envolvidos na recuperação dos arquivos enviados aos servidores.
        
        Para recuperar um arquivo, é de grande importância a capacidade de listar os arquivos de um diretório. Para essa ação é utilizado o protocolo da figura \ref{fig:protListFiles}. Esse protocolo é implementado pela função \textit{list\_files}, que é apresentada na figura \ref{fig:interfaceComunicacao}.
        
        \begin{figure}[!ht]
        \bordaProtocolo{
            $C \rightarrow S: directoryKey$\\
            $S \rightarrow C: metaFile_{0}, metaFile_{1}, metaFile_{2},...,metaFile_{n}$
        }
        \caption{Requisição da lista dos arquivos em um diretório}
        \label{fig:protListFiles}
        \end{figure}
        
        Ao requisitar ao servidor os arquivos do diretório referenciado por $directoryKey$ o servidor manda pacotes de informação $metaFile$ referente aos arquivos. Cada pacote desse é composto por:
        \begin{itemize}
            \item nome do arquivo
            \item tamanho do arquivo
            \item dono do arquivo
            \item data de criação
            \item data de modificação
        \end{itemize}
        
        Essas características do arquivo são enviadas junto com o nome do arquivo para evitar acessos desnecessários ao servidor para recuperar cada uma dessas informações posteriormente. Isso influencia muito no tempo de resposta do sistema.
        
        Com a lista dos arquivos o usuário pode requisitar um arquivo especifico ao servidor. Ao requisitar o arquivo o módulo cliente faz a solicitação do \textit{salt} referente ao arquivo pelo atributo nome, com essa informação gera o trio de chaves do arquivo. Com o identificador do arquivo (\textit{verify key}) o cliente solicita a um servidor uma lista com quais servidores possuem as porções do arquivo. Essa comunicação é formalizada na figura \ref{fig:protWhoHasParts}. Esse protocolo é implementado pela função \textit{who\_has\_parts}, que é apresentada na figura \ref{fig:interfaceComunicacao}.
        
         \begin{figure}[!ht]
        \bordaProtocolo{
            $C \rightarrow S: VK, userID$ \\
            $S \rightarrow C: (S_{0},1),(S_{0},2),(S_{1},1),(S_{1},3),(S_{2},2),(S_{2},3),...$
        }

        \caption{Já com a \textit{verify key}, é solicitado uma lista dos servidores que possuem as partes do arquivo}
        \label{fig:protWhoHasParts}
        \end{figure}
        
        Ao fazer a solicitação de quais servidores possuem as partes do arquivo desejado, o cliente recebe uma lista de registros de quais servidores possuem qual parte do arquivo, no seguinte formato: ($SERVIDOR$,$PARTE DO ARQUIVO$). Esse é o formato utilizado na figura \ref{fig:protWhoHasParts}.
        
        Com a lista dos servidores que possuem as partes do seu arquivo o cliente pode finalmente requisitar as partes do arquivo. Esse processo é feito de acordo com o protocolo apresentado pela figura \ref{fig:protNegociateGetPart} e é implementado pela função \textit{negociate\_get\_parts}, que é apresentada na figura \ref{fig:interfaceComunicacao}.
        
                \begin{figure}[!ht]
        \bordaProtocolo{
            Para $i$ de $1$ até $N$:\\
                $C \rightarrow S_{i}: verifyKey,part{i},ip,port$
        }
        \caption{Módulo cliente envia informação para os servidores, negociando o recebimento das partes}
        \label{fig:protNegociateGetPart}
        \end{figure}
        
        Esse protocolo diz que o cliente é quem irá abrir a conexão para receber o arquivo, e o servidor ira se conectar com o cliente utilizando o $ip$ e $port$ para isso. Após conectado com o cliente o servidor enviará o arquivo para o cliente baseado pelo protocolo apresentado pela figura\ref{fig:protGetPart}. Uma vez que esse protocolo também não é feito via XML-RPC não é definida uma função especifica.
        
        \begin{figure}[!ht]
        \bordaProtocolo{
            Para $i$ de $1$ até $N$:\\
                $S_{i} \rightarrow C: filePart_{i}$
        }
        \caption{Servidores enviando porções do arquivo para o cliente}
        \label{fig:protGetPart}
        \end{figure}
        
        
        
        Uma vez que o cliente tenha todas as partes do arquivo a junção dessas partes é feita e o arquivo é descriptografado utilizando a chave de criptografia \textit{read key} que apenas o cliente possui.
        
        A operação de atualização de um arquivo nos servidores é equivalente a remover o arquivo atual e cadastrar um novo arquivo. Como o protocolo de envio de arquivos já foi apresentado nas figuras \ref{fig:protSendFile} e \ref{fig:protSendFileData}, são feitas a seguir a definição da do protocolo de remoção de um arquivo dos servidores. Esse protocolo é apresentado na figura \ref{fig:protRemoveFile} e é implementado pela função \textit{remove\_file} da figura \ref{fig:interfaceComunicacao}.
        
        \begin{figure}[!ht]
        \bordaProtocolo{
            Para $i$ de $1$ até $N$:
            $C \rightarrow S_{i}: verifyKey,writeKey$
        }
        \caption{Protocolo que define a remoção das partes de um arquivo.}
        \label{fig:protRemoveFile}
        \end{figure}
        
        
        Como definido pelo protocolo de remoção de arquivos, deve fazer a busca dos servidores que possuem as partes do arquivo desejado, utilizando o protocolo já especificado na figura \ref{fig:protWhoHasParts}. Então o cliente envia ao servidor o arquivo que deseja excluir e a chave de escrita no arquivo. O cliente deve fazer isso para todos os servidores. Essa operação é feita no cliente para manter a filosofia do FlexA de trazer a complexidade de processamento e comunicação para o cliente.
        
        Além de enviar, listar, excluir e atualizar os arquivos remotos uma operação que também é de grande importância é a de mover arquivos para outro diretório. No FlexA essa operação é feita totalmente no módulo servidor de forma que o cliente não necessita enviar novamente o arquivo ao move-lo de diretório. Essa operação tem o protocolo de comunicação especificado na figura \ref{fig:protMoveFile}, e é implementado pela função \textit{move\_file} da figura \ref{fig:interfaceComunicacao}.
        
        \begin{figure}[!ht]
        \bordaProtocolo{
            $ C \rightarrow S: verifyKey,writeKey,destinationFileName, directoryKey $
        }
        \caption{Protocolo para mover um arquivo.}
        \label{fig:protMoveFile}
        \end{figure}
        
        
        Outro recurso muito importante do FlexA é o compartilhamento de arquivos, que é implementado através do compartilhamento das chaves de criptografia. Para que seja possível fazer o compartilhamento e armazenar o estado deles o servidor cadastra esses dados criptografados utilizando a chave pública do usuário que recebe a autorização para acessar o arquivo compartilhado. O protocolo de comunicação para a execução desse procedimento é apresentado na figura \ref{fig:protShareFile}, e é implementado pela função \textit{share\_file} da figura \ref{fig:interfaceComunicacao}.
        
        \begin{figure}[!ht]
        \bordaProtocolo{
            $C \rightarrow S: verifyKey, pubKey_{B}, \{accessKeys\}_{pubKey_{B}}$
        }
        \caption{Protocolo para o compartilhamento de arquivos}
        \label{fig:protShareFile}
        \end{figure}
        
        Como mostrado na figura \ref{fig:protShareFile}, o cliente envia as chaves de acesso que desejar ao usuário $B$, cifrando-as com a chave pública de $B$.
        
        
        Para que seja possível executar o procedimento de compartilhamento de arquivos é necessário obter a chave pública de um usuário, esse procedimento é feito através do compartilhamento da chave pública entre os próprios usuários, sem o uso do sistema para isso.
        
        Quando o usuário B desejar requisitar o arquivo que foi compartilhado com ele, o mesmo protocolo de recebimento de arquivos é utilizado, apenas omitindo a parte de solicitação do \textit{salt}, pois já possui as chaves de acesso ao arquivo. Para saber quais arquivos o usuário $B$ tem acesso ele deve listar os arquivos compartilhados cadastrados com sua chave pública, utilizando o protocolo exibido na figura \ref{fig:protListSharedFiles}, que é implementado pela função \textit{list\_shared\_files} da figura \ref{fig:interfaceComunicacao}.
        
        
     \begin{figure}[!ht]
        \bordaProtocolo{
        $C \rightarrow S: userID$
        }
        \caption{Protocolo para o listar os arquivos compartilhados com um usuário.}
        \label{fig:protListSharedFiles}
        \end{figure}
        
        Por fim, após compartilhar um arquivo, caso seja necessário remover o compartilhamento é necessário refazer a criptografia do arquivo, removê-lo do sistema e cadastrá-lo novamente, já que as chaves de acesso foram fornecidas a outro usuário no compartilhamento. Para remover o acesso a um arquivo compartilhado é utilizado o protocolo apresentado na figura \ref{fig:protRemoveShare} que é implementado pela função \textit{unshare\_file} da figura \ref{fig:interfaceComunicacao}.
        
        \begin{figure}[!ht]
        \bordaProtocolo{
            $C \rightarrow S: verifyKey,pubKey_{B}$
        }
        \caption{Remove o acesso do usuário $B$ a um arquivo compartilhado pelo usuário $C$}
        \label{fig:protRemoveShare}
        \end{figure}
        
        Vale ressaltar que embora seja custoso computacionalmente ter que recriptografar os arquivos que foram compartilhados, o FlexA possui bons mecanismos para fazer essa operação, que foram propostos ~\cite{silas} e estudados e implementados por ~\cite{leandro} %% Leandro Gatinho :D huahruahruh S2
        
        Esses são todos os protocolos que coordenam a comunicação entre o módulo cliente e o servidor (e vice e versa) para a manipulação de arquivos. Além desses protocolos ainda existem os que são utilizados para a obtenção dos metadados dos servidores, que são apresentados a seguir.
        
        
        \subsubsection{Protocolos de obtenção de metadados dos servidores}
        
        Para que o cliente possa se comunicar com um servidor ele deve conhece-lo, isso inclui saber seu \textit{serverID}, e o estado atual dos recursos do servidor.
        
        O protocolo que rege a comunicação para a obtenção do ID do servidor é apresentado na figura \ref{fig:protGetServerID} e é implementado pela função \textit{get\_server\_id} da figura \ref{fig:interfaceComunicacao}.
        
        
        \begin{figure}[!ht]
        \bordaProtocolo{
            $S \rightarrow C: serverID$
        }
        \caption{Servidor envia seu \textit{serverID} para cliente.}
        \label{fig:protGetServerID}
        \end{figure}
    
        Além do \textit{serverID} é importante que o servidor tenha um meio de passar informações sobre o estado atual de seus recursos para o cliente. Esse protocolo é de grande importância, pois permite que futuramente possam ser incorporados  os mecanismos de balanceamento de carga propostos e implementados por ~\cite{danilo}.
        O protocolo para a obtenção do estado do servidor é apresentado na figura \ref{fig:protServerGetStatus} e é implementado pela função \textit{get\_server\_status} da figura \ref{fig:interfaceComunicacao}.
        
         \begin{figure}[!ht]
         \bordaProtocolo{
            $S \rightarrow C: status_{1}, status_{2}, status_{3}, ..., status_{n} $
         }
         \caption{Servidor envia para cliente métricas sobre seu estado atual.}
         \label{fig:protServerGetStatus}
         \end{figure}
        
        Com todos esses protocolos definidos e formalizados o sistema possui documentação suficiente para passar para a faze de implementação da versão Android do módulo cliente.
        
        
        
    
    
    
    
    
        \section{Adaptação do servidor em Python}
    
    
        
        
        \section{Módulo Cliente desenvolvido Android}
        
        
        Implementar o módulo cliente para dispositivos móveis que utilizem Android é uma tarefa bem mais complexa que a implementação para interface de linha de comando ou Command-line Interface (CLI) que existe atualmente, pois para se obter um resultado satisfatório é necessário o uso de tarefas assíncronas nas operações de entrada e saída, de modo a evitar o congelamento do aplicativo ao realizar essas tarefas ~\cite{androidAssyncTask}, além da complexidade de exibir as informações utilizando interface gráfica de usuário (GUI).
        
        Dessa forma o foco do desenvolvimento da versão Android do FlexA focou principalmente na conectividade e no uso dos padrões de desenvolvimento estabelecidos com a formalização dos protocolos de comunicação entre cliente e servidor. Como primeiro passo para nortear o desenvolvimento de módulos compatives com o FlexA foi a criação de uma interface de comunicação padrão que garante a compatibilidade entre módulos feitos em linguagens diferentes, desde que utilizado o XML-RPC para a comunicação. Essa interface de comunicação é apresentada na figura \ref{fig:interfaceComunicacao}.
%% Revisado por Gabriel Saraiva

\chapter{Testes, resultados e avaliações}
\label{cap4}

Nesse capítulo serão apresentados os ambientes de testes e os resultados obtidos.

\section{Ambiente de Testes}

    O ambiente de testes utilizou os equipamentos disponíveis no laboratório do GSPD e um celular próprio.
    
    Esses equipamentos são:
    
    \begin{itemize}
    
    
    
    \item \textit{Cluster} heterogêneo com 16 nós:
    
    
        %
        %
        % - - - PEGAR PROCESSO DA FAPESP DO CLUSTER E INSERIR AQUI!!!
        % - - - COMO TEXTO! :)
        %
        %
        
        
    
        \begin{itemize}
        \item 8 nós equipados Intel(R) Core(TM) i7, 16GB de RAM, HD de 500GB
        \item 8 nós com Intel(R) Pentium(R) Dual, 4 GB de RAM, HD de 80GB
        \item Conexão 10/100/1000
        \end{itemize}
        
    \item \textit{Tablet} Samsung GALAXY Note 10.1 - 2014 (referente ao processo FAPESP 2012/02926-5):
        
        \begin{itemize}
            \item Android 4.3 (Jelly Bean)
            \item 3 GB de memória RAM
            \item 16GB de armazenamento interno (\textit{flash})
            \item bateria de 8220 mAh
            \item Chipset Qualcomm Snapdragon 800
            \item CPU Quad-core 1.9 GHz Cortex-A15
            \item GPU Adreno 330
        \end{itemize}
        
    \item \textit{Smart-phone} Motorola Moto X XT1058:
    
        \begin{itemize}
            \item Android  4.4.4 (KitKat)
            \item 2GB de memória RAM
            \item 16GB de armazenamento interno (\textit{flash})
            \item bateria de 2200 mAh
            \item Chipset Qualcomm MSM8960DT Snapdragon S4 Pro
            \item CPU Dual-core 1.7 GHz Krait 300
            \item GPU Adreno 320
            
        \end{itemize}
        
    \item Roteador Wi-fi Asus 300 Mbps
    
    \end{itemize}
    
     Os arquivos utilizados para realizar os testes são fotografias de altíssima resolução obtidas em  ~\cite{hubble}. Sendo uma fotografia de 6.6MB, outra de 15M e a maior de 172MB.
     
     
     Nos testes de desempenho e eficiência, para amenizar influências de outros processos e do sistema operacional, os testes foram realizados 10 vezes com intervalos aleatórios de 1 a 2 minutos. Os tempos utilizados nos gráficos são as médias dessas execuções.

\section{Integridade dos Dados}

Como segundo ~\cite{coulouris}, um dos principais deveres de um sistema de arquivos distribuídos é manter a integridade de seus arquivos, foram feitos testes de integridade dos arquivos, para verificar se o processo de criptografia e transmissão dos arquivos não danificaram os arquivos. 
    
    Inicialmente era calculado o \textit{hash} MD5 dos arquivos, que então eram copiados via cabo de dados para os dispositivos móveis. Então eram transferido e recuperados do FlexA via Wi-fi, e então eram novamente copiados para um computador via cabo de dados. Depois era calculado o \textit{hash} MD5 dos arquivos para mostrar a integridade dos mesmos. Na figura \ref{fig:testesIntegridade} são mostrados os resultados.

    \begin{figure}[!ht]
    \centering
    \includegraphics[width=14cm]{testeIntegridade.png}
    \caption{Teste de integridade dos arquivos com Android. Nenhum arquivo foi danificado durante os testes.}
    \label{fig:testesIntegridade}
    \end{figure}

    Todos os testes mostraram resultados positivos quanto a integridade dos arquivos ao serem enviados e retornarem pelo FlexA utilizando o Android.
    
\section{Compatibilidade do sistema}

    Como o módulo cliente implementado do FlexA deve ser compatível com o módulo existente, foram feitos testes semelhantes aos de integridade mostrado anteriormente, exceto que os arquivos eram inseridos no FlexA pelo cliente em Python e depois recuperados pelo cliente Android. Na figura \ref{fig:testeCompatibilidade} são apresentados os resultados desse teste.
    
    \begin{figure}[!ht]
    \centering
    \includegraphics[width=14cm]{testeCompatibilidade.png}
    \caption{Teste de compatibilidade entre os dois módulos clientes do FlexA. Nenhum arquivo foi danificado durante os testes mostrando que os sistemas estão compatíveis.}
    \label{fig:testeCompatibilidade}
    \end{figure}
    
\section{Desempenho de criptografia}

    Esse teste mostra o tempo necessário para criptografar arquivos os arquivos de testes em cada um dos dispositivos móveis e também em um nó do primeiro grupo do \textit{cluster} (computadores com processador Intel Core i7) para comparação. Os resultados são mostrados na figura \ref{fig:testesCriptografia}.
    
    \begin{figure}[!ht]
    \centering
    \includegraphics[width=14cm]{testeCriptografia.png}
    \caption{Teste de desempenho de criptografia com clientes diferentes do FlexA}
    \label{fig:testesCriptografia}
    \end{figure}
    
    Como esperado, o computador do \textit{cluster} foi capaz de criptografar os arquivos aproximadamente 3,4 vezes mais rápido que o \textit{tablet} e 3,8 vezes mais rápido que o \textit{smart-phone}.
    
\section{Desempenho de transmissão dos dados}

    Esse teste mostra o tempo necessário para transmitir os arquivos via Wi-fi para o FlexA. Os resultados são mostrados na figura \ref{fig:testeTransmissao}.
    
    \begin{figure}[!ht]
    \centering
    \includegraphics[width=14cm]{testeTransmissao.png}
    \caption{Teste de desempenho de transmissão de arquivos com clientes diferentes do FlexA}
    \label{fig:testeTransmissao}
    \end{figure}
    
    Como esperado, o particionamento do arquivo não quase não influenciou na transmissão, uma vez que o gargalo nesse caso é a transmissão sem fio dos dados.
%% Revisado por Gabriel Saraiva

\chapter{Conclusão}
\label{cap5}

    A execução desse projeto mostrou-se muito benéfica ao FlexA como um todo, desde a formalização dos protocolos de comunicação, que tornou o sistema muito mais fácil de ser implementado e entendido, testado e no futuro expandido, até a implementação do cliente Android, que além de oferecer ao usuário outra alternativa mais simples para uso, ainda serviu para aprimorar a flexibilidade do projeto, ao torná-lo mais aberto e padronizado.
    
    Com os testes executados, foi possível perceber principalmente que o sistema está se comportando de forma esperada e que embora o FlexA não seja um sistema de arquivos distribuídos voltado para dispositivos móveis, a segurança dos dados que o sistema oferece, faz com que o tempo total de envio e recebimento dos arquivos via \textit{Wi-fi} seja aceitável. Quanto ao desempenho de criptografia e divisão dos arquivos, os dispositivos móveis são seriamente prejudicados por restrições de autonomia energética, aumentando significativamente o consumo elétrico para os clientes, o que é não é o ideal, mas inevitável dentro das premissas do projeto. Os testes de compatibilidade e integridade mostram que os módulos cliente em Python e em Android estão trabalhando de forma compatível e fornecem os resultados esperados, não danificando os arquivos que trafegam pelo sistema.
    
    Dessa forma, é possível dizer que o projeto atingiu seu objetivo e o trabalho mostrou resultados positivos, principalmente no que se diz respeito a aumento da flexibilidade e abertura do FlexA.
    
    
    \section{Dificuldades}
    
    Os principais desafios para a execução do trabalho foram modelar interfaces concisas para o FlexA e adaptar o módulo servidor, devido a pouca experiência com a linguagem Python. Outro desafio encontrado foi  o aprendizado do uso da base de dados SQLite com a ferramenta de ORM SQLAlchemy.
    
    Deve-se ressaltar também o aprendizado de programação para Android, que apesar de ser feito utilizando linguagem Java, apresenta diversos desafios principalmente relacionados a construção de uma interface gráfica de forma modular e assíncrona.
    
    Por fim, talvez o mais complexo foi a elaboração de interfaces de comunicação que fossem genéricas o suficiente, mas não muito relaxadas, o que fornece ao sistema uma boa flexibilidade.
    
    \section{Trabalhos futuros}
    
    Como sugestões para trabalhos futuros, é altamente recomendado que os protocolos formalizados sejam estudados e adaptados para garantir ao sistema proteção contra ataques simples de repetição e injeção de pacotes, utilizando criptografia de toda a comunicação sensível e uso de \textit{tokens} únicos nas comunicações.
    
    Ainda na sessão de segurança, é de grande interesse a utilização de \textit{hash} para validação da integridade dos arquivos transmitidos de e para os servidores, e também a assinatura digital desses dados. Para a assinatura também é necessário a implementação de um mecanismo de troca de chaves criptográfica entre o servidor e o cliente. Fica sugerido a implementação do algoritmo Diffie-Hellman  para este problema ~\cite{merkle1978secure}.
    
    Por fim, também é sugerido o teste com ferramentas de compressão, para comprimir os dados antes de realizar a criptografia, o que pode acelerar o processo de criptografia e também a a transmissão dos dados.
% ----------------------------------------------------------
% ELEMENTOS PÓS-TEXTUAIS
% ----------------------------------------------------------
\postextual


% ----------------------------------------------------------
% Referências bibliográficas
% ----------------------------------------------------------
\bibliography{bibliografia}

% ----------------------------------------------------------
% Glossário
% ----------------------------------------------------------
%
% Consulte o manual da classe abntex2 para orientações sobre o glossário.
%
%\glossary

% ----------------------------------------------------------
% Apêndices
% ----------------------------------------------------------


\begin{comment}
% ---
% Inicia os apêndices
% ---
\begin{apendicesenv}

% Imprime uma página indicando o início dos apêndices
\partapendices

\begin{appendices}
\chapter{Apendice 1} \label{ap1}


bla
bla
bla
bla

\end{appendices}
% ----------------------------------------------------------

% ----------------------------------------------------------

\end{apendicesenv}
% ---


% ----------------------------------------------------------
% Anexos
% ----------------------------------------------------------

% ---
% Inicia os anexos
% ---
\begin{anexosenv}

% Imprime uma página indicando o início dos anexos
\partanexos

% ---
\input{anexos/an1}
% ---

\end{anexosenv}

\end{comment}
%---------------------------------------------------------------------
% INDICE REMISSIVO
%---------------------------------------------------------------------
\phantompart
%\printindex
%---------------------------------------------------------------------


\end{document}
